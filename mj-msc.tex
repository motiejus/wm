\documentclass[a4paper]{article}

\iffalse
\usepackage[L7x,T1]{fontenc}
\usepackage[lithuanian]{babel}
\else
\usepackage[T1]{fontenc}
\usepackage[english]{babel}
\fi

\usepackage[utf8]{inputenc}
\usepackage{a4wide}
\usepackage{csquotes}
\usepackage[maxbibnames=99,style=authoryear]{biblatex}
\usepackage[pdfusetitle]{hyperref}
\usepackage{enumitem}
\usepackage[toc,page,title]{appendix}
\addbibresource{bib.bib}
\usepackage{caption}
\usepackage{subcaption}
\usepackage{gensymb}
\usepackage{varwidth}
\usepackage{tabularx}
\usepackage{float}
\usepackage{tikz}
\usepackage{minted}
\usetikzlibrary{er,positioning}
\definecolor{mypurple}{RGB}{117,112,179}
\input{version}

\newcommand{\DP}{Douglas \& Peucker}
\newcommand{\VW}{Visvalingam--Whyatt}
\newcommand{\WM}{Wang--M{\"u}ller}

\title{
    Cartographic Generalization of Lines using free software \\
    (example of rivers) \\ \vspace{4mm}
}

\iffalse
\fi

\author{Motiejus Jakštys}

\date{
    \vspace{10mm}
    Version: \VCDescribe \\ \vspace{4mm}
    Generated At: \GeneratedAt
}

\begin{document}
\maketitle

\begin{abstract}
\label{sec:abstract}

Current open-source line generalization solutions have their roots in
    mathematics and geometry, and are not fit for natural objects like rivers
    and coastlines. This paper discusses our implementation of \WM algorithm
    under and open-source license, explains things that we would had
    appreciated in the original paper and compares our results to different
    generalization algorithms.
\end{abstract}

\newpage

\tableofcontents
\listoffigures

\newpage

\section{Introduction}
\label{sec:introduction}

A number of cartographic line generalization algorithms have been researched,
which claim to better process cartographic objects like lines. These fall into
two rough categories:
\begin{itemize}
    \item Cartographic knowledge was encoded to an algorithm (bottom-up
        approach). One among these are \cite{wang1998line}.
    \item Mathematical shape transformation which yields a more
        cartographically suitable down-scaling. E.g. \cite{jiang2003line},
        \cite{dyken2009simultaneous}, \cite{mustafa2006dynamic},
        \cite{nollenburg2008morphing}.
\end{itemize}

During research for the mentioned articles, prototype code has been written for
most of the algorithms. However, none of them seem to be available for use
except for the two "classical" ones -- {\DP} and {\VW}.

\cite{wang1998line} is an algorithm specifically created for cartographic
generalization and available for general use, though it is only currently
available in a commercial product. This poses a problem for map creation in
open source software: there is not a similar high-quality simplification
algorithm to create down-scaled maps, so any cartographic work, which uses line
generalization as part of its processing, will be of sub-par quality.
We believe that availability of high-quality open-source tools is an important
foundation for future cartographic experimentation and development, thus it
it benefits the cartographic society as a whole.

This paper will be reviewing and comparing two widely available algorithms that
are often used for line generalization:
\begin{itemize}
    \item \cite{douglas1973algorithms} via
        \href{https://postgis.net/docs/ST_Simplify.html}{PostGIS Simplify}.

    \item \cite{visvalingam1993line} via
        \href{https://postgis.net/docs/ST_SimplifyVW.html}{PostGIS SimplifyVW}.

\end{itemize}

Since both algorithms produce jaggy output lines, it is worthwhile to process
those through a widely available \cite{chaikin1974algorithm} smoothing
algorithm via \href{https://postgis.net/docs/ST_ChaikinSmoothing.html}{PostGIS
ChaikinSmoothing}.

\section{Visual comparison}

\subsection{Comparison algorithms and parameters}
\subsection{Combining bends}

\section{Conclusions}
\label{sec:conclusions}

\section{Related Work and future suggestions}
\label{sec:related_work}

\printbibliography

\begin{appendices}

\section{Žeimena and Lakaja in context}

\section{Code listings}

For the curious users it may be useful to see how the analysis was executed.
Also, given the source listings, it should be relatively straightforward to
re-run the same analysis on a different area.

The analysis was executed and report was generated on Ubuntu 20.04 with only
system packages. This should be sufficient: {\tt postgis gdal-bin biber
latexmk texlive-bibtex-extra python3-geopandas python3-pygments}.

\subsection{Makefile}
This file binds all the pieces together:
\begin{itemize}
    \item Prepares the PostGIS database.
    \item Generates helper figures (sine waves, squares).
    \item Runs analysis on input files ({\DP}, {\VW}, Chaikin).
    \item Invokes {\tt latexmk} as a final report generation step.
\end{itemize}
\inputminted[fontsize=\small]{make}{Makefile}

\subsection{layer2img.py}
This file accepts a layer (or two) and generates a PDF image suitable for embedding into the report.
\inputminted[fontsize=\small]{python}{layer2img.py}

\subsection{db}
Manages a PostGIS database in the Docker container. That way, the database can
be torn down and re-created by automated tools like the {\tt Makefile} itself.
\inputminted[fontsize=\small]{bash}{db}

\end{appendices}
\end{document}
